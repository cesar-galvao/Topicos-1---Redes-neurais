% Options for packages loaded elsewhere
\PassOptionsToPackage{unicode}{hyperref}
\PassOptionsToPackage{hyphens}{url}
\PassOptionsToPackage{dvipsnames,svgnames,x11names}{xcolor}
%
\documentclass[
  a4paperpaper,
]{article}

\usepackage{amsmath,amssymb}
\usepackage{iftex}
\ifPDFTeX
  \usepackage[T1]{fontenc}
  \usepackage[utf8]{inputenc}
  \usepackage{textcomp} % provide euro and other symbols
\else % if luatex or xetex
  \ifXeTeX
    \usepackage{mathspec} % this also loads fontspec
  \else
    \usepackage{unicode-math} % this also loads fontspec
  \fi
  \defaultfontfeatures{Scale=MatchLowercase}
  \defaultfontfeatures[\rmfamily]{Ligatures=TeX,Scale=1}
\fi
\usepackage{lmodern}
\ifPDFTeX\else  
    % xetex/luatex font selection
\fi
% Use upquote if available, for straight quotes in verbatim environments
\IfFileExists{upquote.sty}{\usepackage{upquote}}{}
\IfFileExists{microtype.sty}{% use microtype if available
  \usepackage[]{microtype}
  \UseMicrotypeSet[protrusion]{basicmath} % disable protrusion for tt fonts
}{}
\makeatletter
\@ifundefined{KOMAClassName}{% if non-KOMA class
  \IfFileExists{parskip.sty}{%
    \usepackage{parskip}
  }{% else
    \setlength{\parindent}{0pt}
    \setlength{\parskip}{6pt plus 2pt minus 1pt}}
}{% if KOMA class
  \KOMAoptions{parskip=half}}
\makeatother
\usepackage{xcolor}
\usepackage[top=30mm,left=30mm,right=30mm,heightrounded]{geometry}
\setlength{\emergencystretch}{3em} % prevent overfull lines
\setcounter{secnumdepth}{5}
% Make \paragraph and \subparagraph free-standing
\ifx\paragraph\undefined\else
  \let\oldparagraph\paragraph
  \renewcommand{\paragraph}[1]{\oldparagraph{#1}\mbox{}}
\fi
\ifx\subparagraph\undefined\else
  \let\oldsubparagraph\subparagraph
  \renewcommand{\subparagraph}[1]{\oldsubparagraph{#1}\mbox{}}
\fi

\usepackage{color}
\usepackage{fancyvrb}
\newcommand{\VerbBar}{|}
\newcommand{\VERB}{\Verb[commandchars=\\\{\}]}
\DefineVerbatimEnvironment{Highlighting}{Verbatim}{commandchars=\\\{\}}
% Add ',fontsize=\small' for more characters per line
\usepackage{framed}
\definecolor{shadecolor}{RGB}{241,243,245}
\newenvironment{Shaded}{\begin{snugshade}}{\end{snugshade}}
\newcommand{\AlertTok}[1]{\textcolor[rgb]{0.68,0.00,0.00}{#1}}
\newcommand{\AnnotationTok}[1]{\textcolor[rgb]{0.37,0.37,0.37}{#1}}
\newcommand{\AttributeTok}[1]{\textcolor[rgb]{0.40,0.45,0.13}{#1}}
\newcommand{\BaseNTok}[1]{\textcolor[rgb]{0.68,0.00,0.00}{#1}}
\newcommand{\BuiltInTok}[1]{\textcolor[rgb]{0.00,0.23,0.31}{#1}}
\newcommand{\CharTok}[1]{\textcolor[rgb]{0.13,0.47,0.30}{#1}}
\newcommand{\CommentTok}[1]{\textcolor[rgb]{0.37,0.37,0.37}{#1}}
\newcommand{\CommentVarTok}[1]{\textcolor[rgb]{0.37,0.37,0.37}{\textit{#1}}}
\newcommand{\ConstantTok}[1]{\textcolor[rgb]{0.56,0.35,0.01}{#1}}
\newcommand{\ControlFlowTok}[1]{\textcolor[rgb]{0.00,0.23,0.31}{#1}}
\newcommand{\DataTypeTok}[1]{\textcolor[rgb]{0.68,0.00,0.00}{#1}}
\newcommand{\DecValTok}[1]{\textcolor[rgb]{0.68,0.00,0.00}{#1}}
\newcommand{\DocumentationTok}[1]{\textcolor[rgb]{0.37,0.37,0.37}{\textit{#1}}}
\newcommand{\ErrorTok}[1]{\textcolor[rgb]{0.68,0.00,0.00}{#1}}
\newcommand{\ExtensionTok}[1]{\textcolor[rgb]{0.00,0.23,0.31}{#1}}
\newcommand{\FloatTok}[1]{\textcolor[rgb]{0.68,0.00,0.00}{#1}}
\newcommand{\FunctionTok}[1]{\textcolor[rgb]{0.28,0.35,0.67}{#1}}
\newcommand{\ImportTok}[1]{\textcolor[rgb]{0.00,0.46,0.62}{#1}}
\newcommand{\InformationTok}[1]{\textcolor[rgb]{0.37,0.37,0.37}{#1}}
\newcommand{\KeywordTok}[1]{\textcolor[rgb]{0.00,0.23,0.31}{#1}}
\newcommand{\NormalTok}[1]{\textcolor[rgb]{0.00,0.23,0.31}{#1}}
\newcommand{\OperatorTok}[1]{\textcolor[rgb]{0.37,0.37,0.37}{#1}}
\newcommand{\OtherTok}[1]{\textcolor[rgb]{0.00,0.23,0.31}{#1}}
\newcommand{\PreprocessorTok}[1]{\textcolor[rgb]{0.68,0.00,0.00}{#1}}
\newcommand{\RegionMarkerTok}[1]{\textcolor[rgb]{0.00,0.23,0.31}{#1}}
\newcommand{\SpecialCharTok}[1]{\textcolor[rgb]{0.37,0.37,0.37}{#1}}
\newcommand{\SpecialStringTok}[1]{\textcolor[rgb]{0.13,0.47,0.30}{#1}}
\newcommand{\StringTok}[1]{\textcolor[rgb]{0.13,0.47,0.30}{#1}}
\newcommand{\VariableTok}[1]{\textcolor[rgb]{0.07,0.07,0.07}{#1}}
\newcommand{\VerbatimStringTok}[1]{\textcolor[rgb]{0.13,0.47,0.30}{#1}}
\newcommand{\WarningTok}[1]{\textcolor[rgb]{0.37,0.37,0.37}{\textit{#1}}}

\providecommand{\tightlist}{%
  \setlength{\itemsep}{0pt}\setlength{\parskip}{0pt}}\usepackage{longtable,booktabs,array}
\usepackage{calc} % for calculating minipage widths
% Correct order of tables after \paragraph or \subparagraph
\usepackage{etoolbox}
\makeatletter
\patchcmd\longtable{\par}{\if@noskipsec\mbox{}\fi\par}{}{}
\makeatother
% Allow footnotes in longtable head/foot
\IfFileExists{footnotehyper.sty}{\usepackage{footnotehyper}}{\usepackage{footnote}}
\makesavenoteenv{longtable}
\usepackage{graphicx}
\makeatletter
\def\maxwidth{\ifdim\Gin@nat@width>\linewidth\linewidth\else\Gin@nat@width\fi}
\def\maxheight{\ifdim\Gin@nat@height>\textheight\textheight\else\Gin@nat@height\fi}
\makeatother
% Scale images if necessary, so that they will not overflow the page
% margins by default, and it is still possible to overwrite the defaults
% using explicit options in \includegraphics[width, height, ...]{}
\setkeys{Gin}{width=\maxwidth,height=\maxheight,keepaspectratio}
% Set default figure placement to htbp
\makeatletter
\def\fps@figure{htbp}
\makeatother

\usepackage{fvextra}
\DefineVerbatimEnvironment{Highlighting}{Verbatim}{breaklines,commandchars=\\\{\}}
\DefineVerbatimEnvironment{OutputCode}{Verbatim}{breaklines,commandchars=\\\{\}}
\makeatletter
\@ifpackageloaded{caption}{}{\usepackage{caption}}
\AtBeginDocument{%
\ifdefined\contentsname
  \renewcommand*\contentsname{Índice}
\else
  \newcommand\contentsname{Índice}
\fi
\ifdefined\listfigurename
  \renewcommand*\listfigurename{Lista de Figuras}
\else
  \newcommand\listfigurename{Lista de Figuras}
\fi
\ifdefined\listtablename
  \renewcommand*\listtablename{Lista de Tabelas}
\else
  \newcommand\listtablename{Lista de Tabelas}
\fi
\ifdefined\figurename
  \renewcommand*\figurename{Figura}
\else
  \newcommand\figurename{Figura}
\fi
\ifdefined\tablename
  \renewcommand*\tablename{Tabela}
\else
  \newcommand\tablename{Tabela}
\fi
}
\@ifpackageloaded{float}{}{\usepackage{float}}
\floatstyle{ruled}
\@ifundefined{c@chapter}{\newfloat{codelisting}{h}{lop}}{\newfloat{codelisting}{h}{lop}[chapter]}
\floatname{codelisting}{Listagem}
\newcommand*\listoflistings{\listof{codelisting}{Lista de Listagens}}
\makeatother
\makeatletter
\makeatother
\makeatletter
\@ifpackageloaded{caption}{}{\usepackage{caption}}
\@ifpackageloaded{subcaption}{}{\usepackage{subcaption}}
\makeatother
\ifLuaTeX
\usepackage[bidi=basic]{babel}
\else
\usepackage[bidi=default]{babel}
\fi
\babelprovide[main,import]{portuguese}
% get rid of language-specific shorthands (see #6817):
\let\LanguageShortHands\languageshorthands
\def\languageshorthands#1{}
\ifLuaTeX
  \usepackage{selnolig}  % disable illegal ligatures
\fi
\usepackage{bookmark}

\IfFileExists{xurl.sty}{\usepackage{xurl}}{} % add URL line breaks if available
\urlstyle{same} % disable monospaced font for URLs
\hypersetup{
  pdftitle={Lista 4: Classificação de imagens.},
  pdfauthor={César A. Galvão - 190011572; João Vitor Vasconcelos - 170126064},
  pdflang={pt},
  colorlinks=true,
  linkcolor={blue},
  filecolor={Maroon},
  citecolor={Blue},
  urlcolor={Blue},
  pdfcreator={LaTeX via pandoc}}

\title{Lista 4: Classificação de imagens.}
\author{César A. Galvão - 190011572 \and João Vitor Vasconcelos -
170126064}
\date{}

\begin{document}
\maketitle

\renewcommand*\contentsname{Índice}
{
\hypersetup{linkcolor=}
\setcounter{tocdepth}{3}
\tableofcontents
}
~

\begin{center}\rule{0.5\linewidth}{0.5pt}\end{center}

~

~

~

~

Boa parte dos resultados dessa lista foram obtidos utilizando o apoio do
ChatGPT. Na ausência, ou na dificuldade em achar, de fontes com
instruções para exercícios similares, o exercício se tornou um
aprendizado orientado por prompts do último modelo disponível. Buscou-se
documentar todos os passos da resolução com o intuito de demonstrar que
não foi feito apenas um processo de copiar e colar os resultados da
plataforma, mas uma tentativa de conciliar seus resultados com as aulas,
materiais disponíveis e, quando possível, a documentação do TensorFlow e
Keras. O atendimento de monitoria também foi essencial para nos ajudar a
explorar mais argumentos das funções e possíveis fontes de problemas na
classificação do conjunto de teste.

~

Repositório do GitHub:
\url{https://github.com/cesar-galvao/Topicos-1---Redes-neurais}

Caderno no Colab:
\url{https://colab.research.google.com/drive/13DnHbWgiF4bZfCUiU9F2Ejv2yDWCqxUB?usp=sharing}

\newpage{}

\section{Características dos dados}\label{caracteruxedsticas-dos-dados}

A lista foi resolvida usando duas plataformas. A parte que diz respeito
à estruturação e compreensão dos dados foi feita no RStudio, enquanto o
treinamento da rede neural foi feito no Google Colab.

Antes de iniciar a construção da rede neural, foi necessário entender
como os dados precisariam estar estruturados para serem utilizados.
Questões como a disposição dos arquivos, informações sobre os
\emph{labels} das imagens, formato e dimensão das imagens, entre outras,
foram levantadas. Isso é importante para que se possa construir um
modelo que seja capaz de lidar com os dados de maneira eficiente.

A seguir, os arquivos são transformados. A justificativa para isso é
descrita a seguir.

\begin{Shaded}
\begin{Highlighting}[]
\NormalTok{pacman}\SpecialCharTok{::}\FunctionTok{p\_load}\NormalTok{(dplyr, magick)}
\end{Highlighting}
\end{Shaded}

\begin{Shaded}
\begin{Highlighting}[]
\CommentTok{\# diretorios das imagens}
\NormalTok{dir\_imagens }\OtherTok{\textless{}{-}} \FunctionTok{c}\NormalTok{(}\StringTok{"../lista 4/Teste/"}\NormalTok{)}

\CommentTok{\# lista de arquivos .bmp}
\NormalTok{bmp }\OtherTok{\textless{}{-}} \FunctionTok{list.files}\NormalTok{(dir\_imagens, }\AttributeTok{pattern =} \StringTok{"}\SpecialCharTok{\textbackslash{}\textbackslash{}}\StringTok{.BMP"}\NormalTok{, }\AttributeTok{full.names =} \ConstantTok{TRUE}\NormalTok{)}

\CommentTok{\# transforma as imagens em png}
\ControlFlowTok{for}\NormalTok{ (file }\ControlFlowTok{in}\NormalTok{ bmp) \{}
\NormalTok{  img }\OtherTok{\textless{}{-}} \FunctionTok{image\_read}\NormalTok{(file)}
  
  \CommentTok{\# cria os paths para escrita}
\NormalTok{  output\_file }\OtherTok{\textless{}{-}} \FunctionTok{file.path}\NormalTok{(dir\_imagens, }\FunctionTok{paste0}\NormalTok{(tools}\SpecialCharTok{::}\FunctionTok{file\_path\_sans\_ext}\NormalTok{(}\FunctionTok{basename}\NormalTok{(file)), }\StringTok{".png"}\NormalTok{))}
  \CommentTok{\# Escreve imagem em png}
  \FunctionTok{image\_write}\NormalTok{(img, }\AttributeTok{path =}\NormalTok{ output\_file, }\AttributeTok{format =} \StringTok{"png"}\NormalTok{)}
\NormalTok{\}}

\CommentTok{\# exclui arquivos bmp}
\FunctionTok{file.remove}\NormalTok{(bmp, }\AttributeTok{showWarnings =} \ConstantTok{FALSE}\NormalTok{)}
\end{Highlighting}
\end{Shaded}

\begin{Shaded}
\begin{Highlighting}[]
\NormalTok{img }\OtherTok{\textless{}{-}} \FunctionTok{image\_read}\NormalTok{(}\StringTok{"../lista 4/Treino/F\_0015.png"}\NormalTok{)}

\FunctionTok{image\_info}\NormalTok{(img)}
\end{Highlighting}
\end{Shaded}

\begin{longtable}[]{@{}lrrllrl@{}}
\toprule\noalign{}
format & width & height & colorspace & matte & filesize & density \\
\midrule\noalign{}
\endhead
\bottomrule\noalign{}
\endlastfoot
PNG & 96 & 103 & Gray & TRUE & 6268 & 38x38 \\
\end{longtable}

\begin{Shaded}
\begin{Highlighting}[]
\FunctionTok{as.integer}\NormalTok{(}\FunctionTok{image\_data}\NormalTok{(img)) }\SpecialCharTok{\%\textgreater{}\%} \FunctionTok{dim}\NormalTok{()}
\end{Highlighting}
\end{Shaded}

\begin{verbatim}
[1] 103  96   1
\end{verbatim}

Observa-se que:

\begin{itemize}
\tightlist
\item
  Há 740 imagens de digitais femininas e 3260 imagens de digitais
  masculinas para treino;
\item
  Há 2000 imagens para teste;
\item
  As imagens estão em formato \texttt{bmp}, que é um formato de imagem
  nativo do Windows. Optou-se por transformá-las em \texttt{png}, que é
  um formato mais comum e que é suportado por mais plataformas;
\item
  As imagens em \texttt{png} possuem dimensão 96x103, com apenas um
  canal em escala de cinza.
\end{itemize}

Quanto à atribuição dos labels de treino, o guia
\href{https://tensorflow.rstudio.com/tutorials/keras/classification}{TensorFlow
for R} foi consultado, porém não houve instruções claras. Para
esclarecimento, foi consultado o ChatGPT, que indicou a utilização de
uma arvore de arquivos para a organização dos dados. A seguinte
estrutura foi sugerida, mas utilizada com adaptações, visto que a função
utilizada inclui uma separação nativa dos dados para validação:

~

\begin{Shaded}
\begin{Highlighting}[]
\ExtensionTok{/path\_to\_dataset/}
    \ExtensionTok{/train/}
        \ExtensionTok{/class1/}
            \ExtensionTok{img1.jpg}
            \ExtensionTok{img2.jpg}
            \ExtensionTok{...}
        \ExtensionTok{/class2/}
            \ExtensionTok{img1.jpg}
            \ExtensionTok{img2.jpg}
            \ExtensionTok{...}
    \ExtensionTok{/validation/}
        \ExtensionTok{/class1/}
            \ExtensionTok{img1.jpg}
            \ExtensionTok{img2.jpg}
            \ExtensionTok{...}
        \ExtensionTok{/class2/}
            \ExtensionTok{img1.jpg}
            \ExtensionTok{img2.jpg}
            \ExtensionTok{...}
\end{Highlighting}
\end{Shaded}

~

Dessa forma, os arquivos foram organizados da seguinte forma:

\begin{itemize}
\tightlist
\item
  Treino

  \begin{itemize}
  \tightlist
  \item
    male
  \item
    female
  \end{itemize}
\item
  Teste
\end{itemize}

~

\section{Ajuste do ambiente}\label{ajuste-do-ambiente}

O ambiente do Colab foi ajustado com os seguintes pacotes e Runtime T4
GPU, visando acelerar o treinamento da rede.

~

\begin{Shaded}
\begin{Highlighting}[]
\FunctionTok{install.packages}\NormalTok{(}\StringTok{"keras"}\NormalTok{)}
\FunctionTok{library}\NormalTok{(keras)}
\FunctionTok{install.packages}\NormalTok{(}\StringTok{"tensorflow"}\NormalTok{)}
\FunctionTok{library}\NormalTok{(tensorflow)}
\FunctionTok{install.packages}\NormalTok{(}\StringTok{"reticulate"}\NormalTok{)}
\FunctionTok{library}\NormalTok{(reticulate)}

\CommentTok{\# GPU}
\NormalTok{tf}\SpecialCharTok{$}\NormalTok{config}\SpecialCharTok{$}\FunctionTok{list\_physical\_devices}\NormalTok{(}\StringTok{"GPU"}\NormalTok{)}
\end{Highlighting}
\end{Shaded}

~

\section{Pré-processamento}\label{pruxe9-processamento}

De acordo com o a documentação do
\href{https://cran.r-project.org/web/packages/keras/keras.pdf}{Keras}, o
pré-processamento via \texttt{image\_data\_generator()} já não é mais
adequado, sendo sugerida uma função mais nova. No entanto, não
identificamos um fluxo de trabalho que utilize, conforme a documentação,
a função \texttt{image\_dataset\_from\_directory()} e as demais camadas
de pré-processamento\footnote{``\emph{image\_data\_generator is not
  recommended for new code. Prefer loading images with
  image\_dataset\_from\_directory and transforming the output TF Dataset
  with preprocessing layers. For more information, see the tutorials for
  loading images and augmenting images, as well as the preprocessing
  layer guide}''.}.

Dessa forma, optamos por utilizar a função antiga. Além disso, pedimos
uma sugestão ao chatGPT de hiperparâmetros para o pré-processamento e
\emph{data augmentation}, de forma que obtivemos o seguinte resultado:

~

\begin{Shaded}
\begin{Highlighting}[]
\CommentTok{\# Data augmentation sugerida}
\NormalTok{datagen }\OtherTok{\textless{}{-}} \FunctionTok{image\_data\_generator}\NormalTok{(}
  \AttributeTok{rescale =} \DecValTok{1}\SpecialCharTok{/}\DecValTok{255}\NormalTok{,               }\CommentTok{\# Rescale pixel values}
  \AttributeTok{validation\_split =} \FloatTok{0.3}\NormalTok{,        }\CommentTok{\# Split the data into training and validation}
  \AttributeTok{rotation\_range =} \DecValTok{20}\NormalTok{,           }\CommentTok{\# Randomly rotate images by up to 20 degrees}
  \AttributeTok{width\_shift\_range =} \FloatTok{0.2}\NormalTok{,       }\CommentTok{\# Randomly shift images horizontally by 20\% of the width}
  \AttributeTok{height\_shift\_range =} \FloatTok{0.2}\NormalTok{,      }\CommentTok{\# Randomly shift images vertically by 20\% of the height}
  \AttributeTok{shear\_range =} \FloatTok{0.2}\NormalTok{,             }\CommentTok{\# Randomly shear images}
  \AttributeTok{zoom\_range =} \FloatTok{0.2}\NormalTok{,              }\CommentTok{\# Randomly zoom in and out}
  \AttributeTok{horizontal\_flip =} \ConstantTok{TRUE}         \CommentTok{\# Randomly flip images horizontally}
\NormalTok{)}
\end{Highlighting}
\end{Shaded}

~

Considerando que estamos tratando de imagens de digitais, algumas dessas
sugestões fazem sentido, enquanto outras não. Os argumentos são
discutidos individualmente a seguir:

\begin{itemize}
\tightlist
\item
  Reescalar os valores dos pixels para o intervalo {[}0, 1{]} faz
  sentido. Imaginamos que isso possa reduzir um problema de escala;
\item
  Utilizaremos 30\% dos dados de treinamento para validação;
\item
  Rotacionar as imagens em até 20 graus faz sentido, visto que as
  digitais podem estar em diferentes orientações no momento de captura
  da imagem;
\item
  A rotação também aparenta fazer sentido, ja que pode existir variação
  no momento de captura;
\item
  Alterações verticais e horizontais (artificialmente neste caso, para
  \emph{data augmentation}) também fazem sentido como forma de
  generalização de formatos de dedos;
\item
  A operação de sheer (distorção da imagem apenas no sentido horizontal)
  não parece fazer sentido. Até poderia fazer sentido caso houvesse uma
  distorção natural ou no processo de digitalização das imagens, mas
  decidimos não incluir;
\item
  O zoom também faz sentido, já que podemos estar tratando de
  generalização de pedaços das imagens;
\item
  Finalmente, a inversão horizontal também faz sentido, já que isso
  possivelmente estaríamos tratando de imagens de mãos direitas e
  esquerdas.
\end{itemize}

~

Outros argumentos disponíveis não parecem fazer sentido de serem
alterados. Por exemplo, não faria sentido suavizar a imagem com
normalização \emph{feature} ou \emph{samplewise}, já que estamos
tratando de imagens com contornos finos e isso possivelmente contribui
para a classificação correta da imagem. Por outro lado,
\texttt{fill\_mode} indica formas de preenchimento de pixels na
fronteira. Para evitar que o comportamento na convolução seja dominado
por \texttt{cval\ =\ 0} por padrão, selecionamos
\texttt{fill\_mode\ =\ "nearest"}, que expande os valores dos pixels
mais próximos.

Além disso, depois de algumas tentativas de calibragem do modelo e
classificação das imagens de teste com esses hiperparâmetros notamos que
algumas configurações geravam resultados melhores com a mesma
arquitetura de rede. O principal resultado era obter todas as
classificações de uma única classe e suspeitamos que isso ocorria devido
ao desbalanceamento da amostra, que foi arrumado utilizando pesos para
as classes, cuja solução é apresentada no código de classificação. Junto
a essa solução, os hiperparâmetros que apresentaram a melhor
classificação foram os seguintes:

~

\begin{Shaded}
\begin{Highlighting}[]
\CommentTok{\# Data augmentation utilizada}
\NormalTok{datagen }\OtherTok{\textless{}{-}} \FunctionTok{image\_data\_generator}\NormalTok{(}
  \AttributeTok{validation\_split =} \FloatTok{0.3}\NormalTok{,}
  \AttributeTok{rotation\_range =} \DecValTok{20}\NormalTok{,}
  \AttributeTok{width\_shift\_range =} \FloatTok{0.2}\NormalTok{,}
  \AttributeTok{height\_shift\_range =} \FloatTok{0.2}\NormalTok{,}
  \AttributeTok{zoom\_range =} \FloatTok{0.2}\NormalTok{,}
  \AttributeTok{horizontal\_flip =} \ConstantTok{TRUE}
\NormalTok{)}
\end{Highlighting}
\end{Shaded}

~

\section{Ajuste da rede neural}\label{ajuste-da-rede-neural}

Definidas as configurações de pré-processamento, a rede neural foi
ajustada novamente com sugestões do ChatGPT com algumas alterações. A
implementação tem os seguintes passos:

\begin{enumerate}
\def\labelenumi{\arabic{enumi}.}
\tightlist
\item
  Carregamento de pacotes;
\item
  Definição do pré-processamento;
\item
  Definição do diretório de dados;
\item
  Definição dos conjuntos de treinamento e de validação;
\item
  Sequenciamento do modelo;
\item
  Compilação do modelo.
\end{enumerate}

O passo 3 foi definido conforme o código a seguir:

~

\begin{Shaded}
\begin{Highlighting}[]
\NormalTok{train\_dir }\OtherTok{\textless{}{-}} \StringTok{"/content/Treino"}
\CommentTok{\# contem as pastas com labels de male e female}
\end{Highlighting}
\end{Shaded}

~

O passo 4 ocorreu como segue. A função recebe o diretório dos dados de
treinamento (com as imagens de homens e mulheres em pastas distintas) e
o objeto com as especificações de pré-processamento.
\texttt{target\_size} é definido para o tamanho das imagens, de modo que
não reverta para o padrão de 256x256 pixels, com cores em escala de
cinza e um batch size de 32, por padrão. Como a classificação de fato é
binária --- homens e mulheres --- faz sentido manter o argumento de
\texttt{class\_mode}. O subset é indicado para utilizar aquele que foi
indicado no pré-processamento. O argumento \texttt{shuffle} não foi
alterado, utilizando \texttt{TRUE} como padrão e embaralhando as imagens
para favorecer uma amostragem probabilística. O argumento
\texttt{classes} foi incluido na mesma ordem de exibição das pastas no
diretório de treino para atribuir classes identificáveis.

~

\begin{Shaded}
\begin{Highlighting}[]
\NormalTok{train\_generator }\OtherTok{\textless{}{-}} \FunctionTok{flow\_images\_from\_directory}\NormalTok{(}
\NormalTok{  train\_dir,}
  \AttributeTok{generator =}\NormalTok{ datagen,}
  \AttributeTok{target\_size =} \FunctionTok{c}\NormalTok{(}\DecValTok{96}\NormalTok{, }\DecValTok{103}\NormalTok{),}
  \AttributeTok{color\_mode =} \StringTok{"grayscale"}\NormalTok{,}
  \AttributeTok{batch\_size =} \DecValTok{32}\NormalTok{,}
  \AttributeTok{class\_mode =} \StringTok{"binary"}\NormalTok{,}
  \AttributeTok{subset =} \StringTok{"training"}\NormalTok{,}
  \AttributeTok{classes =} \FunctionTok{c}\NormalTok{(}\StringTok{\textquotesingle{}female\textquotesingle{}}\NormalTok{, }\StringTok{\textquotesingle{}male\textquotesingle{}}\NormalTok{)}
\NormalTok{)}

\CommentTok{\#image\_Dataset\_From\_directory: labels na ordem}

\NormalTok{validation\_generator }\OtherTok{\textless{}{-}} \FunctionTok{flow\_images\_from\_directory}\NormalTok{(}
\NormalTok{  train\_dir,}
  \AttributeTok{generator =}\NormalTok{ datagen,}
  \AttributeTok{target\_size =} \FunctionTok{c}\NormalTok{(}\DecValTok{96}\NormalTok{, }\DecValTok{103}\NormalTok{),}
  \AttributeTok{color\_mode =} \StringTok{"grayscale"}\NormalTok{,}
  \AttributeTok{batch\_size =} \DecValTok{32}\NormalTok{,}
  \AttributeTok{class\_mode =} \StringTok{"binary"}\NormalTok{,}
  \AttributeTok{subset =} \StringTok{"validation"}\NormalTok{,}
  \AttributeTok{classes =} \FunctionTok{c}\NormalTok{(}\StringTok{\textquotesingle{}female\textquotesingle{}}\NormalTok{, }\StringTok{\textquotesingle{}male\textquotesingle{}}\NormalTok{)}
\NormalTok{)}
\end{Highlighting}
\end{Shaded}

~

As etapas 5 e 6 foram definidas considerando a arquitetura de rede
sugerida pelo ChatGPT, com alterações. A arquitetura sugerida é composta
por três camadas convolucionais, que seguem o exemplo dado em aula. A
quantidade de filtros aumenta na medida em que a imagem vai sendo
reduzida a subprodutos de dimensões menores e é seguida de uma camada
densa com 512 neurônios. Enquanto todas as ativações são ReLU, assumindo
valores zero ou positivos, a última camada de ativação é uma sigmóide,
que faz sentido com a classificação binária.

Considerando as aulas, decidimos incluir duas camadas \emph{fully
connected} ao final, cada uma com 256 neurônios, com drop-out rate de
0.5. A camada de saída com um neurônio parece adequada com a atividade
proposta de classificação.

Não foram feitas alterações na compilação do modelo.

~

\begin{Shaded}
\begin{Highlighting}[]
\NormalTok{model }\OtherTok{\textless{}{-}} \FunctionTok{keras\_model\_sequential}\NormalTok{() }\SpecialCharTok{\%\textgreater{}\%}
  \FunctionTok{layer\_conv\_2d}\NormalTok{(}\AttributeTok{filters =} \DecValTok{32}\NormalTok{, }\AttributeTok{kernel\_size =} \FunctionTok{c}\NormalTok{(}\DecValTok{3}\NormalTok{, }\DecValTok{3}\NormalTok{), }\AttributeTok{activation =} \StringTok{"relu"}\NormalTok{, }\AttributeTok{input\_shape =} \FunctionTok{c}\NormalTok{(}\DecValTok{96}\NormalTok{, }\DecValTok{103}\NormalTok{, }\DecValTok{1}\NormalTok{)) }\SpecialCharTok{\%\textgreater{}\%}
  \FunctionTok{layer\_max\_pooling\_2d}\NormalTok{(}\AttributeTok{pool\_size =} \FunctionTok{c}\NormalTok{(}\DecValTok{2}\NormalTok{, }\DecValTok{2}\NormalTok{)) }\SpecialCharTok{\%\textgreater{}\%}
  \FunctionTok{layer\_conv\_2d}\NormalTok{(}\AttributeTok{filters =} \DecValTok{64}\NormalTok{, }\AttributeTok{kernel\_size =} \FunctionTok{c}\NormalTok{(}\DecValTok{3}\NormalTok{, }\DecValTok{3}\NormalTok{), }\AttributeTok{activation =} \StringTok{"relu"}\NormalTok{) }\SpecialCharTok{\%\textgreater{}\%}
  \FunctionTok{layer\_max\_pooling\_2d}\NormalTok{(}\AttributeTok{pool\_size =} \FunctionTok{c}\NormalTok{(}\DecValTok{2}\NormalTok{, }\DecValTok{2}\NormalTok{)) }\SpecialCharTok{\%\textgreater{}\%}
  \FunctionTok{layer\_conv\_2d}\NormalTok{(}\AttributeTok{filters =} \DecValTok{128}\NormalTok{, }\AttributeTok{kernel\_size =} \FunctionTok{c}\NormalTok{(}\DecValTok{3}\NormalTok{, }\DecValTok{3}\NormalTok{), }\AttributeTok{activation =} \StringTok{"relu"}\NormalTok{) }\SpecialCharTok{\%\textgreater{}\%}
  \FunctionTok{layer\_max\_pooling\_2d}\NormalTok{(}\AttributeTok{pool\_size =} \FunctionTok{c}\NormalTok{(}\DecValTok{2}\NormalTok{, }\DecValTok{2}\NormalTok{)) }\SpecialCharTok{\%\textgreater{}\%}
  \FunctionTok{layer\_flatten}\NormalTok{() }\SpecialCharTok{\%\textgreater{}\%}
  \FunctionTok{layer\_dense}\NormalTok{(}\AttributeTok{units =} \DecValTok{256}\NormalTok{, }\AttributeTok{activation =} \StringTok{"relu"}\NormalTok{) }\SpecialCharTok{\%\textgreater{}\%}
  \FunctionTok{layer\_dropout}\NormalTok{(}\AttributeTok{rate =} \FloatTok{0.5}\NormalTok{) }\SpecialCharTok{\%\textgreater{}\%}
  \FunctionTok{layer\_dense}\NormalTok{(}\AttributeTok{units =} \DecValTok{256}\NormalTok{, }\AttributeTok{activation =} \StringTok{"relu"}\NormalTok{) }\SpecialCharTok{\%\textgreater{}\%}
  \FunctionTok{layer\_dropout}\NormalTok{(}\AttributeTok{rate =} \FloatTok{0.5}\NormalTok{) }\SpecialCharTok{\%\textgreater{}\%}
  \FunctionTok{layer\_dense}\NormalTok{(}\AttributeTok{units =} \DecValTok{1}\NormalTok{, }\AttributeTok{activation =} \StringTok{"sigmoid"}\NormalTok{)}

\CommentTok{\# Compile the model}
\NormalTok{model }\SpecialCharTok{\%\textgreater{}\%} \FunctionTok{compile}\NormalTok{(}
  \AttributeTok{loss =} \StringTok{"binary\_crossentropy"}\NormalTok{,}
  \AttributeTok{optimizer =} \FunctionTok{optimizer\_rmsprop}\NormalTok{(}\AttributeTok{lr =} \FloatTok{0.0001}\NormalTok{),}
  \AttributeTok{metrics =} \FunctionTok{c}\NormalTok{(}\StringTok{"accuracy"}\NormalTok{)}
\NormalTok{)}
\end{Highlighting}
\end{Shaded}

~

Considerando ainda as aulas sobre otimização e sugestões de listas
anteriores, optamos por incluir \emph{early stopping} para evitar
\emph{overfitting} e possivelmente reduzir tempo computacional
infrutífero. O \emph{early stopping} é definido em função dos resultados
da função de perda e o treinamento será interrompido se não houver
melhora em 10 épocas. Em seguida, o modelo será salvo em
\texttt{best\_model.h5} e apenas o melhor modelo será salvo.

~

\begin{Shaded}
\begin{Highlighting}[]
\NormalTok{early\_stopping }\OtherTok{\textless{}{-}} \FunctionTok{callback\_early\_stopping}\NormalTok{(}
  \AttributeTok{monitor =} \StringTok{"val\_loss"}\NormalTok{,}
  \AttributeTok{patience =} \DecValTok{10}\NormalTok{,}
  \AttributeTok{restore\_best\_weights =} \ConstantTok{TRUE}
\NormalTok{)}

\NormalTok{model\_checkpoint }\OtherTok{\textless{}{-}} \FunctionTok{callback\_model\_checkpoint}\NormalTok{(}
  \AttributeTok{filepath =} \StringTok{"best\_model.h5"}\NormalTok{,}
  \AttributeTok{monitor =} \StringTok{"val\_accuracy"}\NormalTok{,}
  \AttributeTok{save\_best\_only =} \ConstantTok{TRUE}
\NormalTok{)}
\end{Highlighting}
\end{Shaded}

~

Por fim, o modelo é treinado, utilizando como argumentos todas as
especificações dadas anteriormente. Os pesos das classes são definidos
em \texttt{class\_weights} para lidar com o desbalanceamento de classes
da amostra.

~

\begin{Shaded}
\begin{Highlighting}[]
\NormalTok{class\_weights }\OtherTok{\textless{}{-}} \FunctionTok{list}\NormalTok{(}\StringTok{\textquotesingle{}0\textquotesingle{}} \OtherTok{=} \DecValTok{3260} \SpecialCharTok{/}\NormalTok{ (}\DecValTok{3260} \SpecialCharTok{+} \DecValTok{740}\NormalTok{), }\StringTok{\textquotesingle{}1\textquotesingle{}} \OtherTok{=} \DecValTok{740} \SpecialCharTok{/}\NormalTok{ (}\DecValTok{3260} \SpecialCharTok{+} \DecValTok{740}\NormalTok{))}

\NormalTok{history }\OtherTok{\textless{}{-}}\NormalTok{ model }\SpecialCharTok{\%\textgreater{}\%} \FunctionTok{fit}\NormalTok{(}
\NormalTok{  train\_generator,}
  \AttributeTok{steps\_per\_epoch =} \FunctionTok{ceiling}\NormalTok{(train\_generator}\SpecialCharTok{$}\NormalTok{n }\SpecialCharTok{/}\NormalTok{ train\_generator}\SpecialCharTok{$}\NormalTok{batch\_size),}
  \AttributeTok{epochs =} \DecValTok{50}\NormalTok{,}
  \AttributeTok{validation\_data =}\NormalTok{ validation\_generator,}
  \AttributeTok{validation\_steps =} \FunctionTok{ceiling}\NormalTok{(validation\_generator}\SpecialCharTok{$}\NormalTok{n }\SpecialCharTok{/}\NormalTok{ validation\_generator}\SpecialCharTok{$}\NormalTok{batch\_size),}
  \AttributeTok{callbacks =} \FunctionTok{list}\NormalTok{(early\_stopping, model\_checkpoint),}
  \AttributeTok{class\_weight =}\NormalTok{ class\_weights}
\NormalTok{)}
\end{Highlighting}
\end{Shaded}

~

\section{Classificação dos dados de
teste}\label{classificauxe7uxe3o-dos-dados-de-teste}

A classificação dos dados de teste foi feita com o melhor modelo gerado
(conforme definições de \emph{callback}). Além disso, foi necessário
indicar que não há label para esses dados, assim como indicar que as
imagens estão em escala de cinza e com as mesmas dimensões das imagens
de treino.

Outro hiperparâmetro que foi ajustado manualmente foi o limiar de
classificação. O limiar que parecia apresentar resultados mais
condizentes com a amostra de treinamento era de 0.7, visto que o
primeiro quartil das probabilidades geradas como resultado era de
aproximadamente 0.69.

O resultado de \texttt{summary(predictions)} gerava a seguintes medidas
de posição:

\begin{quote}
Min. :0.4378\\
1st Qu.:0.6937\\
Median :0.7370\\
Mean :0.7263\\
3rd Qu.:0.7752\\
Max. :0.8715
\end{quote}

~

\begin{Shaded}
\begin{Highlighting}[]
\NormalTok{best\_model }\OtherTok{\textless{}{-}} \FunctionTok{load\_model\_hdf5}\NormalTok{(}\StringTok{"best\_model.h5"}\NormalTok{)}

\NormalTok{test\_images }\OtherTok{\textless{}{-}} \FunctionTok{image\_dataset\_from\_directory}\NormalTok{(}\StringTok{"/content/Teste"}\NormalTok{,}
                                            \AttributeTok{label\_mode =} \ConstantTok{NULL}\NormalTok{,}
                                            \AttributeTok{color\_mode =} \StringTok{"grayscale"}\NormalTok{,}
                                            \AttributeTok{image\_size =} \FunctionTok{c}\NormalTok{(}\DecValTok{96}\NormalTok{, }\DecValTok{103}\NormalTok{))}

\NormalTok{predictions }\OtherTok{\textless{}{-}} \FunctionTok{predict}\NormalTok{(best\_model, test\_images)}

\CommentTok{\# Interpret the predictions}
\NormalTok{predicted\_labels }\OtherTok{\textless{}{-}} \FunctionTok{ifelse}\NormalTok{(predictions }\SpecialCharTok{\textgreater{}} \FloatTok{0.7}\NormalTok{, }\StringTok{"male"}\NormalTok{, }\StringTok{"female"}\NormalTok{)}
\end{Highlighting}
\end{Shaded}

~

Finalmente, os resultados foram salvos em um arquivo \texttt{csv} com o
nome e classificação correspondente.

\begin{Shaded}
\begin{Highlighting}[]
\CommentTok{\# Get the filenames}
\NormalTok{test\_image\_filenames }\OtherTok{\textless{}{-}} \FunctionTok{list.files}\NormalTok{(}\StringTok{"/content/Teste"}\NormalTok{, }\AttributeTok{recursive =} \ConstantTok{TRUE}\NormalTok{, }\AttributeTok{full.names =} \ConstantTok{TRUE}\NormalTok{)}
\NormalTok{test\_image\_filenames }\OtherTok{\textless{}{-}} \FunctionTok{basename}\NormalTok{(test\_image\_filenames)}

\CommentTok{\# Create a data frame with filenames and predicted labels}
\NormalTok{results }\OtherTok{\textless{}{-}} \FunctionTok{data.frame}\NormalTok{(}
  \AttributeTok{Filename =}\NormalTok{ test\_image\_filenames,}
  \AttributeTok{PredictedLabel =}\NormalTok{ predicted\_labels}
\NormalTok{)}

\CommentTok{\# Export results to CSV}
\FunctionTok{write\_csv}\NormalTok{(results, }\StringTok{"predictions\_no\_rescale.csv"}\NormalTok{)}
\end{Highlighting}
\end{Shaded}




\end{document}
